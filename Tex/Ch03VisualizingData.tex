\chapter{Visualizing Data}
There are two primary uses for data visualization:
\begin{itemize}
    \item To explore data
    \item To communicate data
\end{itemize}

\section{Bar Charts}
A bar chart is a good choice when you want to show how some quantity
varies among some discrete set of items.

A bar chart can also be a good choice for plotting histograms of bucketed
numeric values in order to visually explore how the values
are distributed.

Be judicious when using plt.axis. When creating bar charts it is
considered especially bad form for your y-axis not to start at 0, since this is
an easy way to mislead people (\autoref{A chart with a misleading y-axis})

\figures{A chart with a misleading y-axis}

\section{Line Charts}

We can make line charts using plt.plot. These are a
good choice for showing trends.
\section{Scatterplots}
A scatterplot is the right choice for visualizing the relationship between two
paired sets of data.

If you’re scattering comparable variables, you might get a misleading
picture if you let matplotlib choose the scale. If we include a call to \verb|plt.axis("equal")|, the plot (\autoref{User comparable scale in scatterplot}) more
accurately shows that most of the variation occurs on test 2.

\subfigures{A scatterplot with uncomparable axes}{The same scatterplot with equal axes}{User comparable scale in scatterplot}

\section{For Further Exploration}
\begin{itemize}
    \item The \href{https://matplotlib.org/stable/}{matplotlib Gallery} will give you a good idea of the sorts of
          things you can do with matplotlib (and how to do them).
    \item \href{https://seaborn.pydata.org/}{seaborn} is built on top of matplotlib and allows you to easily
          produce prettier (and more complex) visualizations.
    \item \href{https://altair-viz.github.io/}{Altair} is a newer Python library for creating declarative
          visualizations.
    \item \href{https://d3js.org/}{D3.js} is a JavaScript library for producing sophisticated interactive
          visualizations for the web. Although it is not in Python, it is widely
          used, and it is well worth your while to be familiar with it.
    \item \href{https://docs.bokeh.org/en/latest/}{Bokeh} is a library that brings D3-style visualizations into Python.
\end{itemize}





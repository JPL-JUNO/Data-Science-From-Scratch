\chapter{Machine Learning}
Data science is mostly turning business problems into data problems and collecting
data and understanding data and cleaning data and formatting data, after which
machine learning is almost an afterthought.
\section{Modeling}

What is a model? It's simply a specification of a mathematical (or probabilistic) relationship that exists between different variables.

\section{What Is Machine Learning?}

Everyone has her own exact definition, but we'll use machine learning to refer to creating and using models that are learned from data. In other contexts this might be
called predictive modeling or data mining, but we will stick with machine learning.

We'll look at both supervised models (in which there is a set of data labeled with the
correct answers to learn from) and unsupervised models (in which there are no such
labels). There are various other types, like semisupervised (in which only some of the
data are labeled), online (in which the model needs to continuously adjust to newly
arriving data), and reinforcement (in which, after making a series of predictions, the
model gets a signal indicating how well it did) that we won't cover here.

\section{Overfitting and Underfitting}

Clearly, models that are too complex lead to overfitting and don’t generalize well
beyond the data they were trained on. So how do we make sure our models aren’t too
complex? The most fundamental approach involves using different data to train the
model and to test the model.

If the model was overfit to the training data, then it will hopefully perform really
poorly on the (completely separate) test data. Said differently, if it performs well on
the test data, then you can be more confident that it’s fitting rather than overfitting.
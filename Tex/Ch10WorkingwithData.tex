\chapter{Working with Data}
\section{Exploring Your Data}
After you've identified the questions you're trying to answer and have gotten your
hands on some data, you might be tempted to dive in and immediately start building
models and getting answers. But you should resist this urge. Your first step should be
to explore your data.
\subsection{Exploring One-Dimensional Data}
An obvious first step is to compute a few summary statistics. You'd like to know how
many data points you have, the smallest, the largest, the mean, and the standard deviation.

But even these don't necessarily give you a great understanding. A good next step is to
create a histogram, in which you group your data into discrete buckets and count how
many points fall into each bucket.

\subsection{Two Dimensions}
\subsection{Many Dimensions}
With many dimensions, you'd like to know how all the dimensions relate to one
another. A simple approach is to look at the \emph{correlation matrix}, in which the entry in
row $i$ and column $j$ is the correlation between the ith dimension and the jth dimension of the data.


A more visual approach (\notes{if you don’t have too many dimensions}) is to make a scatterplot matrix \autoref{Scatterplot matrix} showing all the pairwise scatterplots.

\figures{Scatterplot matrix}

\section{Using NamedTuples}
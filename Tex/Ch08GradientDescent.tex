\chapter{Gradient Descent}
Frequently when doing data science, we’ll be trying to the find the best model for a
certain situation. And usually “best” will mean something like “minimizes the error
of its predictions” or “maximizes the likelihood of the data.” In other words, it will
represent the solution to some sort of optimization problem.

Our approach will be a technique called \emph{gradient descent}, which lends itself pretty well to a from-scratch treatment.

\section{The Idea Behind Gradient Descent}
The gradient (如果可微) gives the input direction in which the function most quickly increases.

\section{Estimating the Gradient}
If \verb|f| is a function of one variable, its derivative at a point \verb|x| measures how \verb|f(x)|
changes when we make a very small change to \verb|x|.

When \verb|f| is a function of many variables, it has multiple partial derivatives, each indicating how \verb|f| changes when we make small changes in just one of the input variables.

\section{Choosing the Right Step Size}
Although the rationale for moving against the gradient is clear, how far to move is not.
Indeed, choosing the right step size is more of an art than a science. Popular
options include:
\begin{itemize}
    \item Using a fixed step size
    \item Gradually shrinking the step size over time
    \item At each step, choosing the step size that minimizes the value of the objective function
\end{itemize}
\section{Using Gradient Descent to Fit Models}
For the whole dataset we’ll look at the \emph{mean squared error}(MSE). \important{And the gradient of the mean squared error is just the mean
    of the individual gradients.}

\section{Minibatch and Stochastic Gradient Descent}
One drawback of the preceding approach is that we had to evaluate the gradients on the entire dataset before we could take a gradient step and update our parameters. 计算代价太大,尤其是对大型数据集。

We can do this using a technique called \emph{minibatch gradient descent}, in which we compute the gradient (and take a gradient step) based on a “minibatch” sampled from the larger dataset.

\begin{tcolorbox}
    The \verb|TypeVar(T)| allows us to create a “generic” function. It says that
    our dataset can be a list of any single type—strs, ints, lists,
    whatever—but whatever that type is, the outputs will be batches of
    it.
\end{tcolorbox}

Another variation is \emph{stochastic gradient descent}(SGD), in which you take gradient steps based on one training example at a time